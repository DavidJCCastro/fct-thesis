%!TEX root = ../template.tex
%%%%%%%%%%%%%%%%%%%%%%%%%%%%%%%%%%%%%%%%%%%%%%%%%%%%%%%%%%%%%%%%%%%%
%% abstract-pt.tex
%% NOVA thesis document file
%%
%% Abstract in Portuguese
%%%%%%%%%%%%%%%%%%%%%%%%%%%%%%%%%%%%%%%%%%%%%%%%%%%%%%%%%%%%%%%%%%%%

\typeout{NT FILE abstract-pt.tex}%


A purificação de proteínas é vital para estudos de biologia proteica e produção de biofármacos. No entanto, otimizar esses métodos de purificação pode ser demorado devido às variações nas técnicas e etapas do protocolo necessárias para cada nova proteína, que precisam ser determinadas experimentalmente por meio de tentativa e erro.

Este projeto propõe uma abordagem para prever um protocolo de purificação para novas proteínas, com base em informações extraídas de literatura revista por pares. Ao combinar as propriedades físico-químicas das proteínas com informações extraídas de artigos que descrevem protocolos de purificação dessas mesmas proteínas, pretendemos treinar um modelo que possa prever com eficiência uma sequência de etapas de purificação, reduzindo o esforço necessário para purificar novas proteínas.

A primeira etapa deste projeto é criar uma ferramenta de extração de dados que nos permita compilar uma base de dados de proteínas, das suas propriedades físico-químicas e dos seus processos de purificação. Um protótipo inicial foi desenvolvido nesta fase para tentar extrair protocolos de purificação da literatura científica. A segunda etapa é usar os dados que extraímos para treinar um modelo para prever o processo de purificação de uma determinada proteína. Discutimos ambas as etapas neste trabalho, com o seu sucesso medido pela eficiência e exatidão das previsões. Ao atingir o nosso objetivo, seríamos capazes de reduzir custos, resíduos químicos e tempo consumido na descoberta de proteínas de valor agregado.

% Palavras-chave do resumo em Português
% \begin{keywords}
% Palavra-chave 1, Palavra-chave 2, Palavra-chave 3, Palavra-chave 4
% \end{keywords}
\keywords{
  Extração de informação \and
  Processamento de Linguagem Natural \and
  Quimioinformática \and
  Purificação de Proteínas
}
% to add an extra black line
