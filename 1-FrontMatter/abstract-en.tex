%!TEX root = ../template.tex
%%%%%%%%%%%%%%%%%%%%%%%%%%%%%%%%%%%%%%%%%%%%%%%%%%%%%%%%%%%%%%%%%%%%
%% abstract-en.tex
%% NOVA thesis document file
%%
%% Abstract in English
%%%%%%%%%%%%%%%%%%%%%%%%%%%%%%%%%%%%%%%%%%%%%%%%%%%%%%%%%%%%%%%%%%%%

\typeout{NT FILE abstract-en.tex}%

Protein purification is vital for protein biology studies and biopharmaceuticals production. Yet, optimizing these purification methods can be time-consuming because of variations in the techniques and protocol steps necessary for each new protein that have to be determined experimentally through trial and error.

This project proposes an approach to predict a purification protocol for new proteins, based on information extracted from peer-reviewed literature. By combining physico-chemical properties of proteins with information extracted from papers describing purification protocols of those same proteins, we aim to train a model that can efficiently predict a sequence of purification steps, reducing the effort necessary to purify new proteins.

The first step of this project is to create a data extraction tool that allows us to compile a database of proteins, their physico-chemical properties and their purification processes. An initial prototype has been developed in this phase to attempt to extract purification protocols from scientific literature. The second step is using the data we extracted to train a model to predict the purification process of a given protein. We discuss both of these steps in this work, with their success measured by the efficiency and accuracy of the predictions. By achieving our goal, we would be able to lower costs, chemical waste and time consumed in the discovery of added value proteins.

% Abstract keywords
\keywords{
  Information Extraction \and
  Natural Language Processing \and
  Cheminformatics
}
