%!TEX root = ../template.tex
%%%%%%%%%%%%%%%%%%%%%%%%%%%%%%%%%%%%%%%%%%%%%%%%%%%%%%%%%%%%%%%%%%%%
%% abstract-en.tex
%% NOVA thesis document file
%%
%% Abstract in English
%%%%%%%%%%%%%%%%%%%%%%%%%%%%%%%%%%%%%%%%%%%%%%%%%%%%%%%%%%%%%%%%%%%%

\typeout{NT FILE abstract-en.tex}%

Protein purification is vital for protein biology studies, yet optimizing these purification methods can be time-consuming because of variations in the techniques and protocol steps necessary for each new protein that have to be determined experimentally through trial and error. Previous works have focused on creating a database of protein purification conditions and using LLMs to extract relevant information from articles. The development of Protein Language Models. In recent years, several Protein Language Models have been proposed which could also be used to learn useful representations of proteins.

This project proposes an approach to predict a purification protocol for new proteins, based on information extracted from the literature. By combining chemical properties of proteins with information extracted from papers describing purification protocols of those same proteins, we aim to train a model that can do this task efficiently in order to reduce the effort necessary to purify new proteins.

%\begin{verbatim}
%\ntsetup{abstractorder={<LANG_1>,...,<LANG_N>}}
%    \ntsetup{abstractorder={<MAIN_LANG>={<LANG_1>,...,<LANG_N>}}}
%\end{verbatim}
%
%For example, for a main document written in German with abstracts written in German, English and Italian (by this order) use:
%\begin{verbatim}
%    \ntsetup{abstractorder={de={de,en,it}}}
%\end{verbatim}
%
%Concerning its contents, the abstracts should not exceed one page and may answer the following questions (it is essential to adapt to the usual practices of your scientific area):
%
%\begin{enumerate}
%  \item What is the problem?
%  \item Why is this problem interesting/challenging?
%  \item What is the proposed approach/solution/contribution?
%  \item What results (implications/consequences) from the solution?
%\end{enumerate}
%
% Abstract keywords
\keywords{
  One keyword \and
  Another keyword \and
  Yet another keyword \and
  One keyword more \and
  The last keyword
}
