%!TEX root = ../template.tex
%%%%%%%%%%%%%%%%%%%%%%%%%%%%%%%%%%%%%%%%%%%%%%%%%%%%%%%%%%%%%%%%%%%
%% chapter1.tex
%% NOVA thesis document file
%%
%% Chapter with introduction
%%%%%%%%%%%%%%%%%%%%%%%%%%%%%%%%%%%%%%%%%%%%%%%%%%%%%%%%%%%%%%%%%%%

\typeout{NT FILE chapter1.tex}%

\chapter{Introduction}
\label{cha:introduction}

\section{Protein Purification}

Proteins are essential molecules that perform a vast range of critical functions in all living organisms. The ability to isolate these molecules is a fundamental requirement for progress in many scientific and industrial fields, including medical research and the development of new biopharmaceuticals ~\cite{du2022progress}. Because proteins naturally exist within complex biological environments, they must be separated from other cellular components before they can be studied or used effectively, therefore, obtaining a pure sample is a vital step in biotechnology. The speed and success of many scientific advancements depend on the efficiency of this isolation process.

Protein purification is the process of isolating a specific protein of interest from a complex biological mixture, such as a cell lysate or a tissue sample. The objective is to remove all non-protein contaminants and other undesirable proteins while maintaining the biological activity and structural integrity of the target molecule.

The primary challenge in this field lies in the immense diversity of proteins. Every protein possesses a unique combination of physico-chemical properties, including molecular weight, net charge, surface hydrophobicity, and specific binding affinities. Because these characteristics vary significantly even between similar proteins, there is no universal one-size-fits-all protocol. For proteins that do not yet have an established protocol, researchers must rely on a labor-intensive trial and error approach. This involves testing various experimental conditions and chemical buffers, which is both time-consuming and resource-heavy, often becoming a bottleneck in biochemical research.

A common strategy used to simplify this process is the use of affinity tags. Affinity tags are short amino acid sequences or proteins genetically fused to the target molecule that act as standardized "handles", providing a predictable way to bind the protein to a purification resin, allowing for efficient separation based on the tag's known chemical affinity. This does however come at a price, since tags can lead to interference with a protein's function \cite{doi:10.1021/acsomega.7b01598}.

To achieve high levels of purity, laboratory workflows rely almost exclusively on liquid chromatography. Chromatography techniques function by passing a mobile phase containing the protein mixture through a stationary phase. The components of the mixture are separated based on how they interact with the stationary phase (see Figure~\ref{fig:chrom_vizs}):

\begin{itemize}
  \item \textbf{Affinity:} Based on specific biological interactions between the protein and a ligand.
  \item \textbf{Size Exclusion:} Based on the physical dimensions and shape of the molecule.
  \item \textbf{Ion Exchange:} Based on the net surface charge of the protein. 
  \item \textbf{Hydrophobic Interaction:} Based on the distribution of non-polar groups on the protein surface.
\end{itemize}

\begin{figure}[htbp]
    \centering
    \begin{subfigure}[b]{0.45\textwidth}
        \includegraphics[width=\textwidth]{diagrams/Affinity-Chromatography.png}
        \caption{Affinity \cite{chrom_figure_1}}
        \label{fig:sub1}
    \end{subfigure}
    \hfill
    \begin{subfigure}[b]{0.45\textwidth}
        \includegraphics[width=\textwidth]{diagrams/Size-Exclusion-Chromatography.png}
        \caption{Size Exclusion \cite{chrom_figure_2}}
        \label{fig:sub2}
    \end{subfigure}
    
    \vspace{1em}
    
    \begin{subfigure}[b]{0.45\textwidth}
        \includegraphics[width=\textwidth]{diagrams/Ion-Exchange-Chromatography.png}
        \caption{Ion Exchange \cite{chrom_figure_3}}
        \label{fig:sub3}
    \end{subfigure}
    \hfill
    \begin{subfigure}[b]{0.45\textwidth}
        \includegraphics[width=\textwidth]{diagrams/Hydrophobic-Interaction-Chromatography.png}
        \caption{Hydrophobic Interaction \cite{chrom_figure_4}}
        \label{fig:sub4}
    \end{subfigure}
    
    \caption{Visualizations for the Different Types of Chromatography}
    \label{fig:chrom_vizs}
\end{figure}

In practice, a single step is rarely sufficient. A complete purification protocol is a chronological sequence of these techniques, organized to progressively refine the sample. These multi-step sequences form the purification protocols that this project aims to be able to predict.

\section{Motivation}
\label{sec:motivation}

Despite more than a century of methodological advances, protein purification remains a major bottleneck in both basic research and industrial biotechnology \cite{du2022progress}. While expression systems and analytical techniques have become increasingly standardized, the design of purification protocols (particularly for non-tagged proteins) continues to rely heavily on empirical optimization. In practice, this often involves iterative testing of chromatography media, buffer compositions, and elution conditions, guided primarily by expert intuition and prior experience rather than formalized predictive principles.

This trial-and-error paradigm has several limitations. It is time-consuming, costly in terms of reagents and labor, and poorly scalable when applied to large numbers of proteins, such as those emerging from modern genomics and structural biology initiatives, like \textit{AlphaFold} \cite{Jumper2021}, a deep learning model that predicts protein structures.

Recent advances in machine learning, particularly sequence-based modeling and natural language processing, provide an opportunity to address this gap. Large public repositories such as the Protein Data Bank \cite{Berman2000} implicitly encode decades of successful purification efforts, while biomedical literature contains detailed experimental protocols that are now accessible in machine-readable form. We assume that by leveraging these resources we can accelerate the discovery of protein purification protocols using deep learning tools and techniques.

\section{Goal}
\label{sec:goal}

The primary goal of this dissertation is to develop a computational system capable of predicting protein purification protocols directly from protein sequence data and derived physico-chemical properties into an ordered, laboratory-ready purification recipe composed of chromatography steps and associated techniques, reflecting strategies that have been validated in prior experimental work.

To achieve this, the project aims to integrate large-scale data mining from structural databases and the biomedical literature with a Transformer-based model architecture designed for sequential purification protocol prediction. The expected output is not a single optimized protocol, but a plausible and informative starting strategy that can guide experimental design and reduce the search space explored during laboratory optimization.
