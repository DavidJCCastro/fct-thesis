%!TEX root = ../template.tex
%%%%%%%%%%%%%%%%%%%%%%%%%%%%%%%%%%%%%%%%%%%%%%%%%%%%%%%%%%%%%%%%%%%%
%% chapter2.tex
%% NOVA thesis document file
%%
%% Chapter with the template manual
%%%%%%%%%%%%%%%%%%%%%%%%%%%%%%%%%%%%%%%%%%%%%%%%%%%%%%%%%%%%%%%%%%%%

\typeout{NT FILE chapter2.tex}%

\chapter{Related Work}
\label{cha:related_work}


% Começar por extraçao de informaçao e data mining mais geral (incluindo este paper)
% Repositorios de texto como o europe pmc (relevantes para o meu trabalho)
\section{Mining the neuroimaging literature}

% Falar das arquiteturas usadas atualmente
\section{BioBERT}
\section{GliNER-biomed}
\section{Large-Scale Biomedical Relation Extraction Across Diverse Relation Types}
\section{BioT5+: Towards Generalized Biological Understanding with IUPAC Integration and Multi-task Tuning}

% Representação de proteinas
% Explicar bases de dados que existem
% Propriedades quimicas que vamos usar
\section{Protein Representation Learning via Knowledge Enhanced Primary Structure Reasoning}


% Confirmar com o Armenio as propriedades que vamos usar


% Trabalho semelhantes ao que estou a fazer

\section{PurificationDB}

\section{Using Large Language Model to Optimize Protein Purification}